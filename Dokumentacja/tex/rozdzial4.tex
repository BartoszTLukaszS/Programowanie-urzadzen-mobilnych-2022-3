	\newpage
\section{Implementacja}		%4
%Wkleić szkielet kodu, wraz z komentarzami. Opisać zmienne, struktury do czego służą. Opisać procedury, metody co wykonują. Opisać nowe zdefiniowane klasy. Opisać dziedziczenie. Opisać nowo utworzone pliki za co odpowiadają.

\subsection{Wybrane kody z objaśnieniami działania}		%4.1

\subsubsection{Zakładka Historia} %4.1.1

\hspace{0.60cm}Kod \ref{lst:listing-cpp} przedstawia sposób inicjalizacji połączenia z bazą danych SQLite:
\begin{lstlisting}[caption=Połączenie z bazą danych, label={lst:listing-cpp}, language=C++]
	public History()
	{
		InitializeComponent();
		
		_database = new SQLiteAsyncConnection(Path.Combine(Environment.
		GetFolderPath(Environment.SpecialFolder.LocalApplication
		Data), "trainingHistory.db3"));
	}
\end{lstlisting}

Kod \ref{lst:listing-cpp2} przedstawia sposób wypisania danych na ekran:
\begin{lstlisting}[caption=Wypisanie danych na ekran, label={lst:listing-cpp2}, language=C++]
	protected override async void OnAppearing()
	{
		base.OnAppearing();
		
		//Wypisanie danych na ekran
		collectionView.ItemsSource = await GetTrainingData();
	}
\end{lstlisting}

Kod \ref{lst:listing-cpp3} przedstawia sposóB w jaki pobierane są dane z bazy danych:
\begin{lstlisting}[caption=Pobranie danych z bazy, label={lst:listing-cpp3}, language=C++]
	public async Task<List<TrainingData>> GetTrainingData()
	{
		var query = await _database.Table<TrainingData>().ToListAsync();
		results = Enumerable.Reverse(query).ToList();
		return results;
	}
\end{lstlisting}

Kod \ref{lst:listing-cpp4} zawiera funkcję otwieracjącą widok statystyk i podsumowania:
\begin{lstlisting}[caption=Otwieranie widoku statystyk i podsumowania, label={lst:listing-cpp4}, language=C++]
	private async void OpenStatistics(object sender, EventArgs e)
	{
		var btn = (Button)sender;
		await Navigation.PushAsync(new StatisticsView(btn.ClassId));
	}
\end{lstlisting}

Kod \ref{lst:listing-cpp5} przedstawia sposób w jaki dane z bazy danych zostaną wypisane na ekran:
\begin{lstlisting}[caption=Wypisanie danych z bazy danych na ekran, label={lst:listing-cpp5}, language=C++]
	public async Task<List<TrainingData>> GetTrainingData()
	{
		var query = _database.Table<TrainingData>().Where(p => p.Id == buttonId);
		var result =  await query.ToListAsync();
		nazwa.Text = "Trening: " + result[0].DateDay;
		
		_databaseTraining = new SQLiteAsyncConnection(Path.Combine(Environment.GetFolderPath(Environment.SpecialFolder.LocalApplicationData), result[0].TrainingDatabase));
		var locations = await _databaseTraining.Table<CurrentData>().ToListAsync();
		
		Polyline polyline = new Polyline();
		polyline.StrokeColor = Color.FromHex("#192126");
		polyline.StrokeWidth = 20;
		foreach (var pos in locations)
		polyline.Geopath.Add(new Position(pos.Latitude, pos.Longitude));
		map.MapElements.Add(polyline);
		
		Position startPos = new Position(locations[0].Latitude, locations[0].Longitude);
		map.MoveToRegion(MapSpan.FromCenterAndRadius(startPos, Distance.FromKilometers(1)));
		
		
		dataStartu.Text = result[0].DateDay;
		godzinaStartu.Text = result[0].DateTime;
		dystans.Text = result[0].Distance;
		czas.Text = result[0].Time;
		calories.Text = result[0].Calories.ToString() + " cal";
		avrSpeed.Text = result[0].AvrSpeed.ToString() + " km/h";
		return result;
	}
\end{lstlisting}

\subsubsection{Zakładka Trening} %4.1.2

\hspace{0.60cm}Kod \ref{lst:listing-cpp6} przedstawia sposób inicjalizacji połączenia z bazą danych:
\begin{lstlisting}[caption=Połączenie z bazą danych, label={lst:listing-cpp6}, language=C++]
	public Training()
	{
		InitializeComponent();
		
		_database = new SQLiteAsyncConnection(Path.Combine(Environment.
		GetFolderPath(Environment.SpecialFolder.
		LocalApplicationData), "trainingHistory.db3"));
		_database.CreateTableAsync<TrainingData>();
		GetLocation();
	}
\end{lstlisting}

Kod \ref{lst:listing-cpp7} zawiera funkcję pobierająca lokalizację:
\begin{lstlisting}[caption=Pobranie lokalizacji, label={lst:listing-cpp7}, language=C++]
	private async void GetLocation()
	{
		//pobranie lokalizacji
		var request = new GeolocationRequest(GeolocationAccuracy.Best);
		var location = await Geolocation.GetLocationAsync(request);
		
		if (location != null)
		{
			position = new Position(location.Latitude, location.Longitude);
			
			//przesuniecie widoku na lokalizacje przeciwnika
			map.MoveToRegion(MapSpan.FromCenterAndRadius(position, Distance.
			FromKilometers(0.5)));
			
			if (isTraining)
			{   
				//zapis danych lokalizacji
				positionsList.Add(new PositionList { Location = location, TimeLasted = 
					(hours * 3600 + mins * 60 + secs) });
				await SaveCurrentData(new CurrentData
				{
					Latitude = location.Latitude,
					Longitude = location.Longitude,
					TimeLasted = (hours * 3600 + mins * 60 + secs)
				});
				UpdateInfo();
			}
			
		}
		
		//pobranie lokalizacji wywolywane co 2 sekundy
		await Task.Delay(2000);
		GetLocation();
		
	}
\end{lstlisting}

Kod \ref{lst:listing-cpp8} wprowadza możliwość rozpoczęcia treningu:
\begin{lstlisting}[caption=Rozpoczęcie treningu, label={lst:listing-cpp8}, language=C++]
	private void StartTraining(object sender, EventArgs e)
	{
		btnStartF.IsVisible = false;
		btnResumeF.IsVisible = false;
		btnEndF.IsVisible = false;
		btnStopF.IsVisible = true;
		isTraining = true;
		startDate = DateTime.Now;
		
		//baza danych lokalizacji w treningu
		_databaseTraining = new SQLiteAsyncConnection(Path.Combine(Environment.GetFolderPath(Environment.SpecialFolder.LocalApplicationData), "training" + startDate.ToString("dd_MM_yyyy_HH_mm_ss") + ".db3"));
		_databaseTraining.CreateTableAsync<CurrentData>();
		
		//inicjalizacja timera
		timer = new Timer();
		timer.Interval = 1000;
		timer.Elapsed += Timer_Elapsed;
		timer.Start();
	}
\end{lstlisting}

Kod \ref{lst:listing-cpp9} zawiera funkcję liczącą sekundy, wywoływaną co sekundę:
\begin{lstlisting}[caption=Liczenie sekund (co sekundę:), label={lst:listing-cpp9}, language=C++]
	private void Timer_Elapsed(object sender, ElapsedEventArgs e)
	{
		secs++;
		if (secs == 59)
		{
			mins++;
			secs = 0;
		}
		if (mins == 59)
		{
			hours++;
			mins = 0;
		}
		Device.BeginInvokeOnMainThread(() =>
		{
			timerValue.Text = string.Format("{0:00}:{1:00}:{2:00}", hours, mins, secs);
		});
	}
\end{lstlisting}

Kod \ref{lst:listing-cpp10} umożliwia zakończenie treningu oraz obsługę danych po odbytym treningu:
\begin{lstlisting}[caption=Zakończenie treningu\, zapis danych do bazy i ich podsumowanie, label={lst:listing-cpp10}, language=C++]
	private async void EndTraining(object sender, EventArgs e)
	{
		btnResumeF.IsVisible = false;
		btnEndF.IsVisible = false;
		btnStopF.IsVisible = false;
		btnStartF.IsVisible = true;
		
		//zapis danych treningu do glownej bazy danych
		await SaveTrainingData(new TrainingData
		{
			DateDay = startDate.ToString("dd/MM/yyyy"),
			DateTime = startDate.ToString("HH:mm"),
			Time = string.Format("{0:00}:{1:00}", (hours * 60 + mins), secs),
			Distance = way.ToString() + " km",
			Calories = caloriesBurned,
			AvrSpeed = avrSpeed, 
			TrainingDatabase = "training" + startDate.ToString("dd_MM_yyyy_HH_mm_ss") + ".db3"
		});
		
		//otwarcie strony podsumowania treningu
		await Navigation.PushAsync(new StatisticsView(CountTrainings().Result.ToString()));
		
		//wyzerowanie zmiennych pomiarowych
		way = 0;
		tempo = 0;
		caloriesBurned = 0;
		avrSpeed = 0;
		speedSum = 0;
		timerValue.Text = "00:00:00";
		amountDistance.Text = "0";
		amountCalories.Text = "0";
		amountSpeed.Text = "0.0";
		hours = 0;
		mins = 0;
		secs = 0;
		map.MapElements.Clear();
	}
\end{lstlisting}

Kod \ref{lst:listing-cpp11} zawiera funkcję aktualizującą informacje na ekranie:
\begin{lstlisting}[caption=Aktualizowanie informacji na ekranie, label={lst:listing-cpp11}, language=C++]
	 private void UpdateInfo()
	{
		//polyline rysuje trase na mapie
		Polyline polyline = new Polyline();
		polyline.StrokeColor = Color.FromHex("#192126");
		polyline.StrokeWidth = 20;
		int length = positionsList.Count - 1;
		double tmpWay;
		if (length > 1)
		{
			//ustawienie lini na dwie ostanie pozycje
			polyline.Geopath.Add(new Position(positionsList[length - 1].Location.Latitude, positionsList[length - 1].Location.Longitude));
			polyline.Geopath.Add(new Position(positionsList[length].Location.Latitude, positionsList[length].Location.Longitude));
			//wywolanie linii na mapie
			map.MapElements.Add(polyline);
			
			//obliczenia parametrow treningu
			tmpWay = Location.CalculateDistance(positionsList[length].Location, positionsList[length - 1].Location, DistanceUnits.Kilometers);
			way = Math.Round((way + tmpWay), 3);
			tempo = (tmpWay / (positionsList[length].TimeLasted - positionsList[length - 1].TimeLasted)) * 3600;
			tempo = Math.Round(tempo, 1);
			speedSum += tempo * (positionsList[length].TimeLasted - positionsList[length - 1].TimeLasted);
			avrSpeed = speedSum / (positionsList[length].TimeLasted);
			avrSpeed = Math.Round(avrSpeed, 1);
			//kalorie na minute = (MET * 3.5 * waga) / 200
			//MET - ile razy wiecej kalorii spala przy danej aktywnosci w porowaniu do odpoczynku
			caloriesBurned = ((avrSpeed * 3.5 * weight) / 200) * (((double)positionsList[length].TimeLasted) / 60);
			caloriesBurned = Math.Round(caloriesBurned, 1);
		}
		
		//wyswietlenie dystansu w okreslonym formacie zaleznym od dystansu
		if (way < 1)
		{
			amountDistance.Text = (way * 1000).ToString();
			distanceSize.Text = " m";
		}
		else if (way < 10)
		{
			amountDistance.Text = string.Format("{0:0.00}", way);
			distanceSize.Text = " km";
		}
		else if (way < 100)
		{
			amountDistance.Text = string.Format("{0:00.0}", way);
			distanceSize.Text = " km";
		}
		else
		{
			amountDistance.Text = string.Format("{0:000}", way);
			distanceSize.Text = " km";
		}
		
		//wyswietlenie predkosci w okreslonym formacie
		if (tempo < 10)
		amountSpeed.Text = string.Format("{0:0.0}", tempo);
		else
		amountSpeed.Text = string.Format("{0:00}", tempo);
		//wyswietlenie spalonych kalorii
		if (caloriesBurned < 10)
		amountCalories.Text = caloriesBurned.ToString();
		else
		amountCalories.Text = string.Format("{0:00}", caloriesBurned);
	}
\end{lstlisting}

Kod \ref{lst:listing-cpp12} wykonuje zapis danych treningu do bazy danych:
\begin{lstlisting}[caption=Zapis danych do bazy danych, label={lst:listing-cpp12}, language=C++]
	public Task<int> SaveTrainingData(TrainingData trainingData)      
	{
		return _database.InsertAsync(trainingData);
	}
\end{lstlisting}

Kod \ref{lst:listing-cpp13} implementuje modułu Step\_Counter, wykorzystujący czujnik akcelerometr:
\begin{lstlisting}[caption=Krakomierz i akcelerometr, label={lst:listing-cpp13}, language=C++]
	public partial class StepCounter : ContentPage
	{
		//inicjalizacja listy przechowujacej liczbe krokow
		List<double> accData = new List<double>();
		int stepsNumber = 0;
		DateTime czas = DateTime.Now;
		TimeSpan interval = TimeSpan.FromSeconds(0.1);
		
		
		public StepCounter()
		{
			InitializeComponent();
			
			Accelerometer.ReadingChanged += Accelerometer_ReadingChanged;
		}
		
		//implementacja modulu akcelerometra
		void Accelerometer_ReadingChanged(object sender, AccelerometerChangedEventArgs args)
		{
			if (DateTime.Now - czas > interval)
			{
				czas = DateTime.Now;
				var xVal = args.Reading.Acceleration.X * 10;
				var yVal = args.Reading.Acceleration.Y * 10;
				var zVal = args.Reading.Acceleration.Z * 10;
				var accValue = Math.Sqrt(xVal * xVal + yVal * yVal + zVal * zVal) - 10;
				if (accData.Count > 240)
				accData.RemoveAt(0);
				accData.Add(accValue);
				
				//petla liczaca kroki
				if (accData.Count > 1)
				{
					for (int i = 1; i < accData.Count - 1; i++)
					{
						if (accData[i] > 1)
						{
							if (accData[i] > accData[i - 1] && accData[i] > accData[i + 1])
							{
								stepsNumber++;
								accData.Clear();
							}
						}
						
					}
				}
				amountSteps.Text = stepsNumber.ToString();
			}
		}
\end{lstlisting}

Kod \ref{lst:listing-cpp14} zawiera funkcję dającą możliwość rozpoczęcia i zakończenia liczenia kroków:
\begin{lstlisting}[caption=Rozpoczęcia i zakończenie liczenia kroków, label={lst:listing-cpp14}, language=C++]
	void Button_Cliked(object sender, EventArgs e) 
	{
		try
		{
			if (Accelerometer.IsMonitoring)
			{
				Accelerometer.Stop();
				btn.Text = "Start";
			}   
			else
			{
				Accelerometer.Start(SensorSpeed.UI);
				btn.Text = "Stop";
			}
			
		}
		catch (FeatureNotSupportedException fnsEx)
		{
			// Not supported on device
		}
		catch (Exception ex)
		{
			// Something else went wrong
		}
	}
\end{lstlisting}