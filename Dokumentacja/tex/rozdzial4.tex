	\newpage
\section{Implementacja}		%4
%Wkleić szkielet kodu, wraz z komentarzami. Opisać zmienne, struktury do czego służą. Opisać procedury, metody co wykonują. Opisać nowe zdefiniowane klasy. Opisać dziedziczenie. Opisać nowo utworzone pliki za co odpowiadają.

\subsection{Wybrane kody z objaśnieniami działania}		%4.1

\subsubsection{Zakładka Historia} %4.1.1

\hspace{0.60cm}Listing \ref{lst:listing-cpp} przedstawia sposób inicjalizacji połączenia z bazą danych SQLite:
\begin{lstlisting}[caption=Połączenie z bazą danych, label={lst:listing-cpp}, language=C++]
	public History()
	{
		InitializeComponent();		
		_database = new SQLiteAsyncConnection(Path.Combine(Environment.
		GetFolderPath(Environment.SpecialFolder.LocalApplication
		Data), "trainingHistory.db3"));
	}
\end{lstlisting}

Listing \ref{lst:listing-cpp2} przedstawia metodę wypisania danych na ekran:
\begin{lstlisting}[caption=Wypisanie danych na ekran, label={lst:listing-cpp2}, language=C++]
	protected override async void OnAppearing()
	{
		base.OnAppearing();		
		//Wypisanie danych na ekran
		collectionView.ItemsSource = await GetTrainingData();
	}
\end{lstlisting}

Listing \ref{lst:listing-cpp3} przedstawia metodę do pobrania danych z bazy danych:
\begin{lstlisting}[caption=Pobranie danych z bazy, label={lst:listing-cpp3}, language=C++]
	public async Task<List<TrainingData>> GetTrainingData()
	{
		var query = await _database.Table<TrainingData>().ToListAsync();
		results = Enumerable.Reverse(query).ToList();
		return results;
	}
\end{lstlisting}

Listing \ref{lst:listing-cpp4} zawiera funkcję otwieracjącą widok statystyk i podsumowania:
\begin{lstlisting}[caption=Otwieranie widoku statystyk i podsumowania, label={lst:listing-cpp4}, language=C++]
	private async void OpenStatistics(object sender, EventArgs e)
	{
		var btn = (Button)sender;
		await Navigation.PushAsync(new StatisticsView(btn.ClassId));
	}
\end{lstlisting}

Listing \ref{lst:listing-cpp5} przedstawia sposób w jaki dane z bazy danych zostaną wypisane na~ekran. Na początku funkcji nawiązujemy połączenie oraz pobieramy dane z bazy danych. Od lini 9 rysujemy przebytą trasę na mapie. Od lini 19 wypisujemy pobrane dane na ekranie.
\begin{lstlisting}[caption=Wypisanie danych z bazy danych na ekran, label={lst:listing-cpp5}, language=C++]
	public async Task<List<TrainingData>> GetTrainingData()
	{
		var query = _database.Table<TrainingData>().Where(p => p.Id == buttonId);
		var result =  await query.ToListAsync();
		
		_databaseTraining = new SQLiteAsyncConnection(Path.Combine(Environment.GetFolderPath(Environment.SpecialFolder.LocalApplicationData), result[0].TrainingDatabase));
		var locations = await _databaseTraining.Table<CurrentData>().ToListAsync();
		
		Polyline polyline = new Polyline();
		polyline.StrokeColor = Color.FromHex("#192126");
		polyline.StrokeWidth = 20;
		foreach (var pos in locations)
		polyline.Geopath.Add(new Position(pos.Latitude, pos.Longitude));
		map.MapElements.Add(polyline);
		
		Position startPos = new Position(locations[0].Latitude, locations[0].Longitude);
		map.MoveToRegion(MapSpan.FromCenterAndRadius(startPos, Distance.FromKilometers(1)));
		
		nazwa.Text = "Trening: " + result[0].DateDay;
		dataStartu.Text = result[0].DateDay;
		godzinaStartu.Text = result[0].DateTime;
		dystans.Text = result[0].Distance;
		czas.Text = result[0].Time;
		calories.Text = result[0].Calories.ToString() + " cal";
		avrSpeed.Text = result[0].AvrSpeed.ToString() + " km/h";
		return result;
	}
\end{lstlisting}

\subsubsection{Zakładka Trening} %4.1.2

\hspace{0.60cm}Listing \ref{lst:listing-cpp6} przedstawia sposób inicjalizacji połączenia z bazą danych, lub utworzenia jej w przypadku pierwszego uruchomienia.
\begin{lstlisting}[caption=Połączenie z bazą danych, label={lst:listing-cpp6}, language=C++]
	public Training()
	{
		InitializeComponent();
		
		_database = new SQLiteAsyncConnection(Path.Combine(Environment.
		GetFolderPath(Environment.SpecialFolder.
		LocalApplicationData), "trainingHistory.db3"));
		_database.CreateTableAsync<TrainingData>();
		GetLocation();
	}
\end{lstlisting}

Listing \ref{lst:listing-cpp7} zawiera funkcję pobierająca lokalizację. Od lini 4 pobieramy aktualne dane lokalizacji użytkownika. Linia 12 ustawia widok na lokalizację użytkownika. Od~lini 15 zapisujemy pobrane wcześniej dane do bazy danych.
W lini 26 wywołujemy funkcję aktualizującą dane na ekranie. Na końcu po dwóch sekundach wywołujemy znów tą samą metodę.
\begin{lstlisting}[caption=Pobranie lokalizacji, label={lst:listing-cpp7}, language=C++]
	private async void GetLocation()
	{
		//pobranie lokalizacji
		var request = new GeolocationRequest(GeolocationAccuracy.Best);
		var location = await Geolocation.GetLocationAsync(request);
		
		if (location != null)
		{
			position = new Position(location.Latitude, location.Longitude);
			
			//przesuniecie widoku na lokalizacje przeciwnika
			map.MoveToRegion(MapSpan.FromCenterAndRadius(position, Distance.
			FromKilometers(0.5)));
			
			if (isTraining)
			{   
				//zapis danych lokalizacji
				positionsList.Add(new PositionList { Location = location, TimeLasted = 
					(hours * 3600 + mins * 60 + secs) });
				await SaveCurrentData(new CurrentData
				{
					Latitude = location.Latitude,
					Longitude = location.Longitude,
					TimeLasted = (hours * 3600 + mins * 60 + secs)
				});
				UpdateInfo();
			}			
		}
		
		//pobranie lokalizacji wywolywane co 2 sekundy
		await Task.Delay(2000);
		GetLocation();
		
	}
\end{lstlisting}

Listing \ref{lst:listing-cpp8} wprowadza możliwość rozpoczęcia treningu. Na początku funkcji ustawiamy widoczność odpowiedniego przycisku. W lini 11 tworzymy bazę danych dla lokalizacji. Od lini 15 tworzymy timer liczący czas treningu.
\begin{lstlisting}[caption=Rozpoczęcie treningu, label={lst:listing-cpp8}, language=C++]
	private void StartTraining(object sender, EventArgs e)
	{
		btnStartF.IsVisible = false;
		btnResumeF.IsVisible = false;
		btnEndF.IsVisible = false;
		btnStopF.IsVisible = true;
		isTraining = true;
		startDate = DateTime.Now;
		
		//baza danych lokalizacji w treningu
		_databaseTraining = new SQLiteAsyncConnection(Path.Combine(Environment.GetFolderPath(Environment.SpecialFolder.LocalApplicationData), "training" + startDate.ToString("dd_MM_yyyy_HH_mm_ss") + ".db3"));
		_databaseTraining.CreateTableAsync<CurrentData>();
		
		//inicjalizacja timera
		timer = new Timer();
		timer.Interval = 1000;
		timer.Elapsed += Timer_Elapsed;
		timer.Start();
	}
\end{lstlisting}

Listing \ref{lst:listing-cpp9} zawiera funkcję liczącą sekundy, wywoływaną co sekundę:
\begin{lstlisting}[caption=Liczenie sekund (co sekundę:), label={lst:listing-cpp9}, language=C++]
	private void Timer_Elapsed(object sender, ElapsedEventArgs e)
	{
		secs++;
		if (secs == 59)
		{
			mins++;
			secs = 0;
		}
		if (mins == 59)
		{
			hours++;
			mins = 0;
		}
		Device.BeginInvokeOnMainThread(() =>
		{
			timerValue.Text = string.Format("{0:00}:{1:00}:{2:00}", hours, mins, secs);
		});
	}
\end{lstlisting}

Listing \ref{lst:listing-cpp10} umożliwia zakończenie treningu oraz obsługę danych po odbytym treningu. Na początku funkcji ustawiamy widoczność odpowiedniego przycisku. Od lini 9 następuje zapis danych treningu do bazy danych. Na końcu od linni 24 następuje wyzerowanie danych treningu, oraz wyczyszczenie mapy.
\begin{lstlisting}[caption=Zakończenie treningu\, zapis danych do bazy i ich podsumowanie, label={lst:listing-cpp10}, language=C++]
	private async void EndTraining(object sender, EventArgs e)
	{
		btnResumeF.IsVisible = false;
		btnEndF.IsVisible = false;
		btnStopF.IsVisible = false;
		btnStartF.IsVisible = true;
		
		//zapis danych treningu do glownej bazy danych
		await SaveTrainingData(new TrainingData
		{
			DateDay = startDate.ToString("dd/MM/yyyy"),
			DateTime = startDate.ToString("HH:mm"),
			Time = string.Format("{0:00}:{1:00}", (hours * 60 + mins), secs),
			Distance = way.ToString() + " km",
			Calories = caloriesBurned,
			AvrSpeed = avrSpeed, 
			TrainingDatabase = "training" + startDate.ToString("dd_MM_yyyy_HH_mm_ss") + ".db3"
		});
		
		//otwarcie strony podsumowania treningu
		await Navigation.PushAsync(new StatisticsView(CountTrainings().Result.ToString()));
		
		//wyzerowanie zmiennych pomiarowych
		way = 0;
		tempo = 0;
		caloriesBurned = 0;
		avrSpeed = 0;
		speedSum = 0;
		timerValue.Text = "00:00:00";
		amountDistance.Text = "0";
		amountCalories.Text = "0";
		amountSpeed.Text = "0.0";
		hours = 0;
		mins = 0;
		secs = 0;
		map.MapElements.Clear();
	}
\end{lstlisting}

Listing \ref{lst:listing-cpp11} zawiera funkcję aktualizującą informacje na ekranie. Funkcja zaczyna się od narysowanie trasy na mapie. Odbywa się to przez narysowanie lini pomiędzy dwoma ostatnimi lokalizacjami użytkowanika. Od lini 18 następuje wyliczenie danych pomiarowych takich jak droga, prędkość, kalorie. Od lini 32 wypisujemy drogę w formacie odpowiednim do wartości przebytej trasy. Od lini 54 wypisujemy prędkość w formacie odpowiednim do wartości aktualnej prędkości. Od lini 59 wypisujemy kalorie w formacie odpowiednim do wartości spalonych kalorii. 
\begin{lstlisting}[caption=Aktualizowanie informacji na ekranie, label={lst:listing-cpp11}, language=C++]
	 private void UpdateInfo()
	{
		//polyline rysuje trase na mapie
		Polyline polyline = new Polyline();
		polyline.StrokeColor = Color.FromHex("#192126");
		polyline.StrokeWidth = 20;
		int length = positionsList.Count - 1;
		double tmpWay;
		if (length > 1)
		{
			//ustawienie lini na dwie ostanie pozycje
			polyline.Geopath.Add(new Position(positionsList[length - 1].Location.Latitude, positionsList[length - 1].Location.Longitude));
			polyline.Geopath.Add(new Position(positionsList[length].Location.Latitude, positionsList[length].Location.Longitude));
			//wywolanie linii na mapie
			map.MapElements.Add(polyline);
			
			//obliczenia parametrow treningu
			tmpWay = Location.CalculateDistance(positionsList[length].Location, positionsList[length - 1].Location, DistanceUnits.Kilometers);
			way = Math.Round((way + tmpWay), 3);
			tempo = (tmpWay / (positionsList[length].TimeLasted - positionsList[length - 1].TimeLasted)) * 3600;
			tempo = Math.Round(tempo, 1);
			speedSum += tempo * (positionsList[length].TimeLasted - positionsList[length - 1].TimeLasted);
			avrSpeed = speedSum / (positionsList[length].TimeLasted);
			avrSpeed = Math.Round(avrSpeed, 1);
			//kalorie na minute = (MET * 3.5 * waga) / 200
			//MET - ile razy wiecej kalorii spala przy danej aktywnosci w porowaniu do odpoczynku
			caloriesBurned = ((avrSpeed * 3.5 * weight) / 200) * (((double)positionsList[length].TimeLasted) / 60);
			caloriesBurned = Math.Round(caloriesBurned, 1);
		}
		
		//wyswietlenie dystansu w okreslonym formacie zaleznym od dystansu
		if (way < 1)
		{
			amountDistance.Text = (way * 1000).ToString();
			distanceSize.Text = " m";
		}
		else if (way < 10)
		{
			amountDistance.Text = string.Format("{0:0.00}", way);
			distanceSize.Text = " km";
		}
		else if (way < 100)
		{
			amountDistance.Text = string.Format("{0:00.0}", way);
			distanceSize.Text = " km";
		}
		else
		{
			amountDistance.Text = string.Format("{0:000}", way);
			distanceSize.Text = " km";
		}
		
		//wyswietlenie predkosci w okreslonym formacie
		if (tempo < 10)
		amountSpeed.Text = string.Format("{0:0.0}", tempo);
		else
		amountSpeed.Text = string.Format("{0:00}", tempo);
		//wyswietlenie spalonych kalorii
		if (caloriesBurned < 10)
		amountCalories.Text = caloriesBurned.ToString();
		else
		amountCalories.Text = string.Format("{0:00}", caloriesBurned);
	}
\end{lstlisting}

Listing \ref{lst:listing-cpp12} wykonuje zapis danych treningu do bazy danych:
\begin{lstlisting}[caption=Zapis danych do bazy danych, label={lst:listing-cpp12}, language=C++]
	public Task<int> SaveTrainingData(TrainingData trainingData)      
	{
		return _database.InsertAsync(trainingData);
	}
\end{lstlisting}

\subsubsection{Zakładka Kroki} %4.1.2

\hspace{0.60cm}Listing \ref{lst:listing-cpp13} implementuje moduł Step\_Counter, wykorzystujący czujnik akcelerometr. W lini 14 przypisujemy funkcję która ma być wywołana kiedy akcelerometr odczyta, że nastapiła zmiana. Na początku metody Accelerometer\_ReadingChanged zapisujemy dane z akcelerometra oraz wyliczamy wektor przesunięcia korzystając z~twierdzenia Pitagorasa. Od lini 32 następuje sprawdzenie warunków koniecznych od ustalenia czy zaszedł krok.
\begin{lstlisting}[caption=Krakomierz i akcelerometr, label={lst:listing-cpp13}, language=C++]
	public partial class StepCounter : ContentPage
	{
		//inicjalizacja listy przechowujacej liczbe krokow
		List<double> accData = new List<double>();
		int stepsNumber = 0;
		DateTime czas = DateTime.Now;
		TimeSpan interval = TimeSpan.FromSeconds(0.1);
		
		
		public StepCounter()
		{
			InitializeComponent();
			
			Accelerometer.ReadingChanged += Accelerometer_ReadingChanged;
		}
		
		//implementacja modulu akcelerometra
		void Accelerometer_ReadingChanged(object sender, AccelerometerChangedEventArgs args)
		{
			if (DateTime.Now - czas > interval)
			{
				czas = DateTime.Now;
				var xVal = args.Reading.Acceleration.X * 10;
				var yVal = args.Reading.Acceleration.Y * 10;
				var zVal = args.Reading.Acceleration.Z * 10;
				var accValue = Math.Sqrt(xVal * xVal + yVal * yVal + zVal * zVal) - 10;
				if (accData.Count > 240)
				accData.RemoveAt(0);
				accData.Add(accValue);
				
				//petla liczaca kroki
				if (accData.Count > 1)
				{
					for (int i = 1; i < accData.Count - 1; i++)
					{
						if (accData[i] > 1)
						{
							if (accData[i] > accData[i - 1] && accData[i] > accData[i + 1])
							{
								stepsNumber++;
								accData.Clear();
							}
						}
						
					}
				}
				amountSteps.Text = stepsNumber.ToString();
			}
		}
\end{lstlisting}

Listing \ref{lst:listing-cpp14} zawiera funkcję dającą możliwość rozpoczęcia i zakończenia liczenia kroków. W zależności od tego czy akcelerometr działa po naciśnięciu przycisku będzie on zatrzymany albo uruchomiony.
\begin{lstlisting}[caption=Rozpoczęcia i zakończenie liczenia kroków, label={lst:listing-cpp14}, language=C++]
	void Button_Cliked(object sender, EventArgs e) 
	{
		try
		{
			if (Accelerometer.IsMonitoring)
			{
				Accelerometer.Stop();
				btn.Text = "Start";
			}   
			else
			{
				Accelerometer.Start(SensorSpeed.UI);
				btn.Text = "Stop";
			}
			
		}
		catch (FeatureNotSupportedException fnsEx)
		{
			// Not supported on device
		}
		catch (Exception ex)
		{
			// Something else went wrong
		}
	}
\end{lstlisting}

\subsubsection{Zakładka Ustawienia} %4.1.3

\hspace{0.60cm}Listing \ref{lst:listing-cpp15} jest to funkcja wywołana po kliknięciu w wybraną opcję. Otwiera ona popup w którym użytkownik może wprowadzić własne prarametry dla danej opcji. Po zatwierdzeniu opcji zostają one przypisane dla określonego prarametru ustawień aplikacji i są zapamiętane.
\begin{lstlisting}[caption=Otwarcie popup do wprowadzenia danych, label={lst:listing-cpp15}, language=C++]
	public async void Open_Popup(object sender, EventArgs e)
	{
		var option = (Xamarin.Forms.Button)sender;
		int id = Int16.Parse(option.ClassId);
		string result = await DisplayPromptAsync(options[id], "");
		Preferences.Set(options[id], result);
		Display_Options();
	}
\end{lstlisting}

Listing \ref{lst:listing-cpp16} jest to funkcja służąca do ustawienia powiadomienia o treningu. Wywołana jest ona po kliknięciu w odpowiednią opcję. Otwiera ona popup w którym użytkownik wprowadza godzinę o której chciałby otrzymać powiadomienie. Następnie od 14 linii następuje ustawienie zmiennej liczącej czas do treningu. Od linii 18 ustawiane są dane powiadomiena. W linii 30 wywołujemy funkcję tworzącą powiadomienie ustawione wcześniej. 
\begin{lstlisting}[caption=Ustawienie powiadomienia o treningu, label={lst:listing-cpp16}, language=C++]
	public async void Set_Notification(object sender, EventArgs e)
	{
		int hour;
		bool valid = false;
		do {
			string result = await DisplayPromptAsync("Podaj godzine. 0-23", "");
			if (result == null)
			result = "18";
			hour = Int16.Parse(result);
			if(hour <= 23 && hour >= 0)
			valid = true;
		} while (!valid);
			
		DateTime setTime = new DateTime(DateTime.Now.Year, DateTime.Now.Month, DateTime.Now.Day, hour, 0, 0);
		int seconds = (setTime.Hour - DateTime.Now.Hour) * 3600 - DateTime.Now.Minute * 60 + DateTime.Now.Second;
		Console.WriteLine(seconds);
		
		var notification = new NotificationRequest
		{
			BadgeNumber = 1,
			Description = "Wyglada, ze zapomniales o treningu",
			Title = "Runly",
			NotificationId = 1337,
			Schedule =
			{
				NotifyTime = DateTime.Now.AddSeconds(seconds),
				RepeatType = NotificationRepeat.Daily,            
			}
		};
		LocalNotificationCenter.Current.Show(notification);
	}
\end{lstlisting}

Listing \ref{lst:listing-cpp17} jest to funkcja, która pobiera z preferencji aplikacji parametry opcji i wyświetla je przy odpowiedniej opcji na ekranie. W przypadku braku ustawienia wartości dla danej opcji wcześniej przez użytkownika wyświetlana jest wartość domyślna opcji. 
\begin{lstlisting}[caption=Wyświetlenie ustawionych opcji w widoku, label={lst:listing-cpp17}, language=C++]
	public void Display_Options()
	{
		Login.Text = Preferences.Get(options[0], "User");
		Waga.Text = Preferences.Get(options[2], "70");
		Wiek.Text = Preferences.Get(options[3], "20");
	}
\end{lstlisting}