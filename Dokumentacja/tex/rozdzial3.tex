	\newpage
\section{Projektowanie}		%3
%Opis przygotowania narzędzi (git, visual studio). Wybór i opis bibliotek, klas. Szkic layoutów. Pseudo kody. Opisy wykorzystanych algorytmów (np. algorytm sortowania). Dokładniejsze określenie założeń i działania aplikacji, (np.: ten przycisk otworzy takie okno a w tym oknie wpisujemy takie dane).

\subsection{Przygotowanie narzędzi (Git, Visual Studio)}		%3.1

\subsubsection{Visual Studio} %3.1.1
\hspace{0.60cm}Pierwszym i oczywistym krokiem, który musimy wykonać jest przygotowanie właściwych technologii i środowisk, które posłużą do tworzenia naszego projektu. Jako środowisko programistyczne wybrano Visual Studio 2022 i dostępną na nim wspomnianą wcześniej platformę Xamarin.Forms. W celu instalacji tego narzędzia, pobieramy plik instalacyjny odpowiedni dla naszego systemu ze strony Microsoftu\footnote{Plik instalacyjny na stronie  https://visualstudio.microsoft.com/pl/vs\cite{www3}.}, co pokazano na rysunku \ref{rys:rysunek002a}. Podczas instalacji, w menedżerze pakietów, wybieramy pakiety \textit{Mobile i desktop developement with ASP.NET}, (rys. \ref{rys:rysunek002b}), zaznaczając też opcjonalne dodatki w zależności od wymagań (rys. \ref{rys:rysunek002c})(np. Xamarin, emulator Androida itd.). Środowisko Visual Studio posiada zintegrowany system kontroli wersji oprogramowania Git, który będzie wspomagał naszą pracę.

\begin{figure}[!htb]
	\centering
	\includegraphics[width=0.8\linewidth]{rys/vs1.png}
	\caption{Pobranie pliku instalacyjnego}
	\label{rys:rysunek002a}
\end{figure}

\begin{figure}[!htb]
	\centering
	\includegraphics[width=0.8\linewidth]{rys/vs2.png}
	\caption{Wybór pakietów}
	\label{rys:rysunek002b}
\end{figure}

\begin{figure}[!htb]
	\centering
	\includegraphics[width=0.8\linewidth]{rys/vs3.png}
	\caption{Dodatkowe pakiety i narzędzia}
	\label{rys:rysunek002c}
\end{figure}

\subsubsection{Git} %3.1.2

\hspace{0.60cm}Na stronie Github stworzone zostało zdalne repozytorium, do którego wysyłane będą kolejne wersje naszego projektu, z możliwością kontrybucji ze strony członków zespołu programistów. Lokalnie, na maszynach programistów zainstalowano dodatkowe narzędzie jakim jest desktopowa aplikacja Git for Windows, która dostarcza emulację BASH wykorzystywaną do uruchamiania Git'a z wiersza poleceń. Jest to duże udogodnienie dla osób, które czują się pewniej pracując z terminalem systemu Linux. Plik instalacyjny można pobrać ze strony\footnote{Plik instalacyjny na stronie https://git-scm.com/download/win\cite{www4}.}.

\subsection{Szkice layoutów i pseudokody}		%3.2

\subsubsection{Zakładka Kroki} %3.2.1

\hspace{0.60cm}W zakładce Kroki rys. \ref{rys:rysunek001a} (s. \pageref{rys:rysunek001a}) wyświetalana zostaje aktualna, obliczana w czasie rzeczywistym liczba odbytych kroków od czasu rozpoczęcia treningu. Początkowo liczba kroków równa się 0. Aplikacja liczy kroki od momentu nasiśnięcia przycisku \textbf{Start} w zakładce Trening (s. \pageref{rys:rysunek001b}). Liczenie kroków dokonuje się z wykorzystaniem modułu akcelerometra. Akcelerometr to wbudowany komponent do pomiaru przyspieszenia dowolnego urządzenia mobilnego. Ruchy takie jak kołysanie, przechylanie, obracanie, drżenie są wykrywane za pomocą akcelerometru. Wartość XYZ służy do obliczania i wykrywania ruchów. Współrzędne X, Y oraz Z reprezentują kierunek i~położenie urządzenia wg. tych osi, w którym nastąpiło przyspieszenie. Dodatkowo przyspieszenie urządzenia obejmuje również przyspieszenie ziemskie (g = 9,81m/s2), które również należy uwzględnić w celu dokłądnego pomiaru.

Obliczanie ogólnego przyspieszenia urządzenia polega na pobieraniu z sensora wartości odpowiednio X, Y oraz Z. Następnie wartości te podstawiane są do odpowiedniego wzoru $\sqrt{(X*X + Y*Y + Z*Z) - 9,81}$, który wykorzystuje też wartość przyspieszenia ziemskiego. Zliczanie kolejnych kroków, począwszy od wartości 0, wykonuje się przy pomocy instrukcji warunkowej if oraz pętli for. Dodatkowo, celu niwelacji niedokładności w pomiarach, zliczane są one dopiero po odbyciu 6 kroków. Listing \ref{lst:listing-ps1} przedstawia za pomocą pseudokodu ogólną zasadę obliczania przyspieszenia:

\begin{lstlisting}[caption=Pseudokod działania krokomierza, label={lst:listing-ps1}, language=C++]
	Lista Zawierajaca_Wartosci_Przyspieszen;
	zmienna X = Odczyt.Sensora.X;
	zmienna Y = Odczyt.Sensora.Y;
	zmienna Z = Odczyt.Sensora.Z;
	
	zmienna Przyspieszenie = sqrt{(X*X + Y*Y + Z*Z) - 9,81};
	
	JESLI (liczba_Wartosci_Przyspieszen > 1)
	{
		pomiar_krokow = 0;
		DLA (i = 1; i < liczba_Wartosci_Przyspieszen - 1; i++)
		{
			pomiar_krokow++;
			JESLI (liczba_krokow > 5)
			{
				liczba_krokow += 6;
			}
		}
	}
\end{lstlisting}

\begin{figure}[!htb]
	\centering
	\includegraphics[width=.2\linewidth]{rys/kroki2.jpg}
	\caption{Layout krokomierza}
	\label{rys:rysunek001a}
\end{figure}

\subsubsection{Zakładka Trening} %3.2.2

\hspace{0.60cm}W zakładce Trening rys. \ref{rys:rysunek001b} (s. \pageref{rys:rysunek001b}) użytkownik ma możliwość rozpoczęcia treningu. Naciśnięcie przycisku \textbf{Start} zaczyna odliczanie i wyświetlanie na ekranie bieżącego czasu odbywanego treningu, pokonany dystans w metrach, bieżącą prędkość oraz teoretyczną ilość spalonych kalorii w zależności od wagi użytkownika. Listing \ref{lst:listing-ps2} przedstawia pseudokod algorytmu wg którego zostaje odliczany czas:

\begin{lstlisting}[caption=Odliczanie czasu, label={lst:listing-ps2}, language=C++]
	klasa Timer;
	int godziny, minuty, sekundy;
	JESLI(trening)
		czas treningu = godziny * 3600 + minuty * 60 + sekundy; 
	Rozpoczecie_Treningu() {
		nowy Timer;
		Inicjalizacja_Timera;
		start Timera;
	}
	Liczenie_czasu_co_sekunde() {
		sekundy++;
		JESLI(sekundy = 59)
			minuty++, sekundy = 0;
		JESLI(minuty = 59)
			godziny++, minuty = 0;
	}
\end{lstlisting}
Dodatkowo wymagane jest załączenie biblioteki System.Timers.

Listing \ref{lst:listing-ps3} przedstawia pseudokod liczenia, w odpowiednim formacie, pokonanego dystansu, prędkości oraz spalonych kalorii. Zmienna \textit{tmpWay} przechowuje dystans w kilometrach pomiędzy dwiema lokalizacjami - lokalizacją obecną i poprzednią. Ten dystans jest obliczany przy użyciu metody \textit{CalculateDistance} w klasie \textit{Location}, która bierze trzy argumenty: obecną lokalizację, poprzednią lokalizację oraz jednostki dystansu. Zmienna \textit{way} przechowuje całkowity dystans. Wartość zmiennej \textit{tmpWay} jest dodawana do \textit{way} i wynik jest zaokrąglany do trzech miejsc po przecinku, używając metody \textit{Math.Round}. Zmienna \textit{tempo} przechowuje wartość prędkości z jaką pokonano drogę z jednej lokalizacji do drugiej. Prędkość jest obliczana poprzez dzielenie dystansu (\textit{tmpWay}) i czasu jaki zajęło jego pokonanie, a następnie wynik mnożony jest przez 3600 w celu uzyskania prędkości w jednostkach na godzinę.

\begin{lstlisting}[caption=Obliczanie parametrów treningu, label={lst:listing-ps3}, language=C++]
	double way = 0;
	double tmpWay;
	double tempo = 0;
	suma_predkosci = 0;
	srednia_predkosc = 0;
	spalone_kalorie = 0;
	double srednia_predkosc = 0;
	double tmpWay = Location.CalculateDistance(lokalizacja_obecna, lokalizacja_poprzednia, dystans);
	way = Math.Round(way + tmpWay, 3);
	tempo = (tmpWay / czas * 3600);
	suma_predkosci += tempo * czas;
	srednia_predkosc = suma_predkosci / czas_calkowity;
	spalone_kalorie = (srednia_predkosc * 3,5 * waga / 200) / (czas_calkowity / 60)
\end{lstlisting}

W celu obliczania sum prędkości, zmienna \textit{tempo} mnożona jest przez czas jaki upłynął między dwiema pozycjami i dodawany do całkowitej sumy prędkości. W~końcu prędkość średnia obliczana jest jako iloraz sum prędkości i całkowitego czasu jaki upłynął. Spalone kalorie obliczane są używając średniej prędkości, wagi użytkownika i całkowitego czasu jaki upłynął.\\ Wzór jaki wykorzystano to: $\dfrac{srednia\_predkosc*3,5*waga}{200} * \dfrac{czas\_calkowity}{60}$.

Na środkowej części ekranu wyświetlana zostaje, przy wykorzystaniu modułów GPS oraz Map Google bieżąca lokalizacja użytkownika. Za pomocą polilinii rysowana jest na mapie trasa, którą udaje się użytkownik. W każdej chwili od rozpoczęcia treningu użytkownik ma możliwość wstrzymania treningu za pomocą przycisku \textbf{Stop}, jak pokazano na rys. \ref{rys:rysunek001ba} (s. \pageref{rys:rysunek001ba}). Po jego kliknięciu, wstrzymane zostaje odliczanie wspomnianych wyżej parametrów.

\begin{figure}[!htb]
	\centering
	\begin{minipage}{.5\textwidth}
		\centering
		\includegraphics[width=.4\linewidth]{rys/trening1.jpg}
		\caption{Layout treningu - rozpoczęcie treningu}
		\label{rys:rysunek001b}
	\end{minipage}%
	\begin{minipage}{.5\textwidth}
		\centering
		\includegraphics[width=.4\linewidth]{rys/trening1a.jpg}
		\caption{Wstrzymanie treningu}
		\label{rys:rysunek001ba}
	\end{minipage}
\end{figure}

\begin{figure}[!htb]
	\centering
	\begin{minipage}{.5\textwidth}
		\centering
		\includegraphics[width=.4\linewidth]{rys/trening1b.jpg}
		\caption{Wznowienie lub zakończenie treningu}
		\label{rys:rysunek001bb}
	\end{minipage}%
	\begin{minipage}{.5\textwidth}
		\centering
		\includegraphics[width=.4\linewidth]{rys/trening2.png}
		\caption{Ekran podsumowania}
		\label{rys:rysunek001c}
	\end{minipage}
\end{figure}

Następnie, jak pokazano na kolejnym rys. \ref{rys:rysunek001bb} (s. \pageref{rys:rysunek001bb}), pojawiają się przyciski \textbf{Wznów} i \textbf{Koniec}. Po kliknięciu tego pierwszego następuje wznowienie odliczania parametrów treningu. Kliknięcie przycisku \textbf{Koniec} kończy cały trening, a tej samej zakładce wyświetlony zostaje ekran zawierający podsumowanie odbytego treningu. Rys. \ref{rys:rysunek001c} (s. \pageref{rys:rysunek001c}) zawiera podsumowanie: datę i czas rozpoczęcia treningu, przebyty dystans, czas treningu, spalone kalorie oraz średnią prędkość z jaką poruszał się użytkownik. 

\subsubsection{Zakładka Historia} %3.2.3

\hspace{0.60cm}W zakładce Historia, po zakończeniu kolejnych treningów, do bazy danych zapisywane są rekordy zawierające dane dotyczące odbytych w przeszłości treningów wraz z podstawywymi danymi jak data, czas oraz statystyki, co pokazano na rys. \ref{rys:rysunek001d} (s. \pageref{rys:rysunek001d}). Dane po zakończeniu treningu są najpierw zapisywane do bazy danych SQLite, następnie wyświetlane w zakładce Historia od najnowszego do najstarszego.

\begin{figure}[!htb]
	\centering
	\includegraphics[width=.2\linewidth]{rys/historia2.jpg}
	\caption{Layout historii treningów}
	\label{rys:rysunek001d}
\end{figure}

Wykorzystana w nim zostaje odpowiednia funkcja otwierająca widok statystyk i podsumowania.

\subsubsection{Zakładka Ustawienia} %3.2.4

\hspace{0.60cm}W zakładce Ustawienia rys. \ref{rys:rysunek001e} (s. \pageref{rys:rysunek001e}) użytkownik ma możliwość konfiguracji własnego konta użytkownika, swoich danych fizycznych oraz wieku, na podstawie których obliczane zostają statystyki traningu jak np. spalone kalorie.

\begin{figure}[!htb]
	\centering
	\includegraphics[width=.2\linewidth]{rys/ustawienia2.jpg}
	\caption{Layout ustawień i opcji}
	\label{rys:rysunek001e}
\end{figure}