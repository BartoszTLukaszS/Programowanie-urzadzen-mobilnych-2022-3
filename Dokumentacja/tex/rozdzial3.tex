	\newpage
\section{Projektowanie}		%3
%Opis przygotowania narzędzi (git, visual studio). Wybór i opis bibliotek, klas. Szkic layoutów. Pseudo kody. Opisy wykorzystanych algorytmów (np. algorytm sortowania). Dokładniejsze określenie założeń i działania aplikacji, (np.: ten przycisk otworzy takie okno a w tym oknie wpisujemy takie dane).

\subsection{Przygotowanie narzędzi (Git, Visual Studio)}		%3.1

\hspace{0.60cm}Pierwszym i oczywistym krokiem, który musimy wykonać jest przygotowanie właściwych technologii i środowisk, które posłużą do tworzenia naszego projektu. Jako środowisko programistyczne wybrano Visual Studio 2022 i dostępną na nim wspomnianą wcześniej platformę Xamarin.Forms. W celu instalacji tego narzędzia, pobieramy plik instalacyjny ze strony Microsoftu. Podczas instalacji, w menedżerze dodatkowych pakietów, wybieramy pakiet \textit{Mobile developement with .NET}, zaznaczając też opcjonalne dodatki w zależności od wymagań (np. Xamarin, emulator Androida itd.). Środowisko Visual Studio posiada zintegrowany system kontroli wersji oprogramowania Git, który będzie wspomagał naszą pracę. 

\hspace{0.60cm}Na stronie Github stworzone zostało zdalne repozytorium, do którego wysyłane będą kolejne wersje naszego projektu, z możliwością kontrybucji ze strony członków zespołu programistów. 

\subsection{Wybrane kody i pseudokody z objaśnieniami działania}		%3.2

\subsubsection{Zakładka Historia} %3.2.1

\hspace{0.60cm}Inicjalizacja połączenia z bazą danych:
\begin{verbatim}
	public History()
	{
		InitializeComponent();
		
		_database = new SQLiteAsyncConnection(Path.Combine(Environment.
		GetFolderPath(Environment.SpecialFolder.LocalApplication
		Data), "trainingHistory.db3"));
	}
\end{verbatim}

\hspace{0.60cm}Wypisanie danych na ekran:
\begin{verbatim}
	protected override async void OnAppearing()
	{
		base.OnAppearing();
		
		//Wypisanie danych na ekran
		collectionView.ItemsSource = await GetTrainingData();
	}
\end{verbatim}

\hspace{0.60cm}Pobranie danych z bazy danych:
\begin{verbatim}
	public async Task<List<TrainingData>> GetTrainingData()
	{
		var query = await _database.Table<TrainingData>().ToListAsync();
		results = Enumerable.Reverse(query).ToList();
		return results;
	}
\end{verbatim}

\hspace{0.60cm}Funkcja otwieracjąca widok statystyk, podsumowania:
\begin{verbatim}
	private async void OpenStatistics(object sender, EventArgs e)
	{
		var btn = (Button)sender;
		await Navigation.PushAsync(new StatisticsView(btn.ClassId));
	}
\end{verbatim}

\subsubsection{Zakładka Trening} %3.2.2

\hspace{0.60cm}Połączenie z bazą danych:
\begin{verbatim}
	public Training()
	{
		InitializeComponent();
		
		//Połączenie z bazą danych
		_database = new SQLiteAsyncConnection(Path.Combine(Environment.
		GetFolderPath(Environment.SpecialFolder.
		LocalApplicationData), "trainingHistory.db3"));
		_database.CreateTableAsync<TrainingData>();
		GetLocation();
	}
\end{verbatim}

\hspace{0.60cm}Funkcja pobierająca lokalizację:
\begin{verbatim}
	private async void GetLocation()
	{
		//Pobranie lokalizacji
		var request = new GeolocationRequest(GeolocationAccuracy.Best);
		var location = await Geolocation.GetLocationAsync(request);
		
		if (location != null)
		{
			position = new Position(location.Latitude, location.Longitude);
			
			//Przesunięcie widoku na lokalizację przeciwnika
			map.MoveToRegion(MapSpan.FromCenterAndRadius(position, Distance.
			FromKilometers(0.5)));
			
			if (isTraining)
			{   
				//Zapis danych lokalizacji
				positionsList.Add(new PositionList { Location = location, TimeLasted = 
					(hours * 3600 + mins * 60 + secs) });
				await SaveCurrentData(new CurrentData
				{
					Latitude = location.Latitude,
					Longitude = location.Longitude,
					TimeLasted = (hours * 3600 + mins * 60 + secs)
				});
				UpdateInfo();
			}
			
		}
		
		//Pobranie lokalizacji wywoływane co 2 sekundy
		await Task.Delay(2000);
		GetLocation();
		
	}
\end{verbatim}

\hspace{0.60cm}Wyświetlenie dystansu w określonym formacie zależnym od dystansu:
\begin{verbatim}
	if (way < 1)
	{
		amountDistance.Text = (way * 1000).ToString();
		distanceSize.Text = " m";
	}
	else if (way < 10)
	{
		amountDistance.Text = string.Format("{0:0.00}", way);
		distanceSize.Text = " km";
	}
	else if (way < 100)
	{
		amountDistance.Text = string.Format("{0:00.0}", way);
		distanceSize.Text = " km";
	}
	else
	{
		amountDistance.Text = string.Format("{0:000}", way);
		distanceSize.Text = " km";
	}
\end{verbatim}

\hspace{0.60cm}Zapisanie danych do bazy danych:
\begin{verbatim}
	public Task<int> SaveTrainingData(TrainingData trainingData)      
	{
		return _database.InsertAsync(trainingData);
	}
	
	public Task<int> CountTrainings()
	{
		return _database.Table<TrainingData>().CountAsync();
	}
	
	public Task<int> SaveCurrentData(CurrentData currentPosition)
	{
		return _databaseTraining.InsertAsync(currentPosition);
	}
\end{verbatim}
